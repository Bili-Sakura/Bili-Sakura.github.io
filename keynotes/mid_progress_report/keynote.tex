\documentclass[12pt,a4paper]{article}

% Use fontspec for font selection (requires LuaLaTeX or XeLaTeX)
\usepackage{fontspec}

\usepackage{xeCJK} % For Chinese support with XeLaTeX

% Set main and other fonts for better Chinese support
\setmainfont{Times New Roman}
\setsansfont{Arial}
\setmonofont{Consolas}

% Set Chinese fonts (adjust names as needed for your system)
\setCJKmainfont{SimSun} % 宋体
% \setCJKsansfont{SimHei} % 黑体
% \setCJKmonofont{FangSong} % 仿宋


% Page geometry
\usepackage[a4paper,margin=2.5cm]{geometry}


% Bibliography (biblatex with biber backend)
\usepackage[backend=biber,style=authoryear]{biblatex}
\addbibresource{references.bib}

% Math and graphics
\usepackage{amsmath, amssymb}
\usepackage{graphicx}
\usepackage{caption}
\usepackage{subcaption}

% Hyperlinks
\usepackage[colorlinks=true,linkcolor=blue,citecolor=blue,urlcolor=blue]{hyperref}

% Section formatting
\usepackage{titlesec}
\titleformat{\section}{\large\bfseries}{\thesection}{1em}{}
\titleformat{\subsection}{\normalsize\bfseries}{\thesubsection}{1em}{}

% Line spacing
\usepackage{setspace}
\onehalfspacing

% Title information
\title{中期进展报告 - 基于扩散模型的高效灾害遥感图像文本控制生成和应用}
\author{陈振源}
\date{\today}

\begin{document}

\maketitle

\begin{abstract}
  遥感图像对灾害应急响应、调度、管理和决策具有重大作用。而灾害发生具有随机性和稀有性,同时,不同地区面临各种灾害的实际情况差异大,在它地的实践经验难以有效的迁移。针对高分辨率时序遥感图像数据集稀少且时空分布不均的问题,本研究创新性的提出一种遥感图像合成方法,使用生成式模型(扩散模型)进行灾害遥感图像基于文本和灾前影像的可控生成,以弥补世界范围内灾害影像数据缺少的问题。
\end{abstract}

\section{背景}
遥感图像在灾害应急响应、调度、管理和决策中发挥着至关重要的作用。通过对灾害发生区域的遥感监测,可以为应急部门提供及时、准确的信息支持,提升灾害响应的效率和科学性。然而,灾害事件具有高度的随机性和稀有性,这使得相关遥感数据的获取变得十分困难。许多灾害发生时无法及时获取高质量的遥感影像,导致数据积累缓慢,难以满足研究和实际应用的需求。此外,不同地区面临的灾害类型和实际情况存在较大差异,导致某一地区的经验和数据难以直接迁移和应用到其他地区。这种区域差异性进一步加剧了数据利用的局限性。目前,高分辨率时序遥感图像数据集不仅数量稀少,而且在时空分布上极为不均衡。部分地区数据丰富,而大多数受灾频发区域却缺乏足够的遥感影像资料。因此,亟需弥补全球范围内灾害遥感影像数据的缺口,推动相关数据的生成与共享,以支持灾害管理和科学研究的深入发展。

传统的遥感图像分析方法在灾害场景下往往难以满足需求。无论是人工还是基于模型的方法,都难以应对灾害的多样性和复杂性;文本引导的编辑能够实现自适应、用户驱动的图像修改,但现有方法大多针对自然场景设计,难以生成真实且语义一致的遥感影像。为此,本工作聚焦于利用高分辨率、双时相RSCC卫星数据,实现灾害场景下的遥感图像文本引导编辑。我们的目标是通过用户指令实现灵活、精准的图像编辑,支持场景推演、快速制图、灾害损失评估和变化检测等灾害管理任务。然而,在遥感领域实现高质量、可控的图像编辑面临技术难题,主要包括领域泛化能力不足以及现有框架的高计算资源需求。当前主流框架在灾害任务中缺乏真实感和语义对齐能力,传统架构计算开销大,难以灵活部署于实际场景。针对上述挑战,我们提出了一种新颖的视觉模型及适应策略。主要贡献包括:面向灾害场景和文本指令的扩散Transformer架构,实现任务引导的遥感图像编辑(该架构专为灾害场景和文本化用户指令定制);高效的模型微调策略,支持大规模适应(显著减少新任务所需的可训练参数,提高适应效率);构建了全面的基准和新的评测指标,专用于文本引导的遥感图像编辑任务(填补了该领域的评测空白);并且在高分辨率、双时相RSCC灾害影像上,相较现有方法取得了实证性提升(在极具挑战性的场景下展现出显著性能提升)。


\section{相关工作}

\subsection{遥感领域的文本到图像生成}

近年来,遥感领域的文本到图像生成方法主要基于扩散模型,通常通过对预训练模型进行微调来实现。在数据有限的场景下,已有研究探索了高效的微调策略~\parencite{ouMethodEfficientSynthesizing2023}。CRS-Diff~\parencite{tangCRSDiffControllableRemote2024} 通过引入 ControlNet~\parencite{controlnet2023} 实现了多模态控制,DiffusionSat~\parencite{khannaDiffusionSatGenerativeFoundation2023} 则进一步提出了 3DControlNet 统一框架,支持时序生成、超分辨率和修复等多种下游任务。Text2Earth~\parencite{liuText2EarthUnlockingTextdriven2025a} 以文本到图像生成为主,同时支持图像编辑和修复等辅助任务。然而,这些方法大多依赖于容量较小的图像主干网络——通常为约 8 亿参数的预训练 UNet(如 Stable Diffusion),与新兴的大型扩散 Transformer(Diffusion Transformer, DiT) 模型相比,其表达能力受到限制。此外,这些方法的文本编码器通常沿用 Stable Diffusion 的 CLIP 变体,对于复杂任务或指令的深层语义理解能力有限。

\subsection{基于扩散模型的图像编辑}

图像编辑是计算机视觉中的基础任务,旨在根据用户意图对给定图像进行修改,同时保持图像的真实感和语义一致性。早期的指令引导图像编辑方法主要依赖于基于 UNet 的扩散模型。值得注意的是,InstructPix2Pix~\parencite{instructpix2pix2023} 首次提出了基于用户指令的编辑新范式,MagicBrush~\parencite{magicbrush2023} 进一步采用该框架,并通过增强数据集提升了性能。

近年来,Diffusion Transformer~\parencite{peebles2023} 的出现为图像合成带来了比传统 UNet 扩散模型更具扩展性和灵活性的架构。在此基础上,StableFlow~\parencite{stableflow2024} 和 RF-Solver~\parencite{rfsolver2024} 等工作提出了无需训练的编辑方法,分别通过关键层注入和注意力机制操控来利用DiT的内部结构,尽管 RF-Solver 并不依赖于标准的指令编辑方式。

与此同时,将多模态大语言模型集成到扩散框架中(如 Qwen2VL-Flux~\parencite{qwen2vlflux2025}、Step1X-Edit~\parencite{step1xedit2025} 和 SmartEdit~\parencite{smartedit2024})使得基于复杂指令的图像编辑更加高效和多样化。另一方面,UltraEdit~\parencite{ultraedit2024} 和 Emu Edit~\parencite{emuedit2024} 仍然采用 UNet 架构,但结合了先进的文本编码器和大规模数据集,实现了更细粒度的指令编辑能力。



\section{方法}
本研究采用了与ICEdit~\parencite{zhangInContextEditEnabling2025}相同的框架。该方法基于大规模DiT,通过上下文生成机制实现指令式图像编辑。具体而言,模型输入包括原始图像、编辑指令以及若干对参考编辑样例(即“上下文对”),模型通过学习这些上下文对之间的编辑关系,理解并执行用户给定的新编辑指令,从而生成符合要求的编辑结果。

在训练阶段,模型接收一组(原图、编辑指令、目标图)三元组,通过条件扩散过程学习从原图到目标图的编辑映射。推理时,用户只需提供原图和新的编辑指令,模型即可在无需额外微调的情况下,基于上下文推理能力完成相应的图像编辑任务。

该方法的核心优势在于:
\begin{itemize}
    \item \textbf{通用性强}:无需针对每种编辑任务单独训练,具备良好的泛化能力。
    \item \textbf{高效的上下文理解}:通过上下文示例,模型能够理解复杂、多样的编辑指令。
    \item \textbf{端到端训练}:整体框架可端到端优化,提升编辑质量和一致性。
\end{itemize}

因此,我们在本项目中直接采用了该框架,利用其强大的指令理解与图像编辑能力,完成了各类复杂的图像编辑任务。

\section{数据集}
本工作使用了RSCC数据集。该数据集面向灾害事件,包含大规模的遥感变化描述,专为灾害感知的双时相遥感影像理解设计。通过视觉推理模型QvQ-Max,对62,351对灾前与灾后影像进行了详细的变化描述标注~\parencite{chenRSCC2025}。

\section{评估指标}

本研究采用多种定量和定性评估指标,全面衡量遥感图像生成与编辑的质量与准确性:

\begin{itemize}
    \item \textbf{CLIP图像相似度}~\parencite{radfordLearningTransferableVisual2021,hesselCLIPScoreReferencefreeEvaluation2021}:利用CLIP模型提取生成图像与目标图像的特征,通过计算特征余弦相似度,评估生成结果与真实目标在高层语义空间中的一致性。
    \item \textbf{DINO图像相似度}~\parencite{oquabDINOv2LearningRobust2024}:采用DINO自监督视觉Transformer模型提取图像特征,计算生成图像与目标图像的特征相似度,进一步衡量结构和内容的一致性。
    \item \textbf{CLIP文图相似度}~\parencite{radfordLearningTransferableVisual2021}:将用户编辑指令与生成图像分别输入CLIP模型,计算文本与图像之间的相似度分数,反映生成结果对指令的语义遵循程度。
    \item \textbf{基于多模态大语言模型的评价指标,如VIEScore}~\parencite{kuVIEScoreExplainableMetrics2024}:引入VIEScore等多模态大语言模型,综合评估图像生成质量和编辑准确性。该类指标能够理解复杂的编辑指令,并对生成图像的语义、内容和编辑效果进行端到端的自动化评价,提升评测的客观性和全面性。
\end{itemize}

通过上述多维度指标,能够系统地评估模型在遥感灾害图像生成与编辑任务中的表现,确保生成结果在视觉质量、语义一致性和指令遵循性等方面均达到高水平。


\section{预期成果}

本项目预计将取得以下几方面的成果:

\begin{itemize}
    \item \textbf{高质量的遥感灾害图像合成与编辑能力}:基于大规模扩散Transformer和上下文编辑机制,能够根据用户文本指令和灾前影像,生成真实感强、语义一致的灾害遥感图像,满足灾害推演、损失评估和变化检测等多样化需求。
    \item \textbf{通用且高效的指令式图像编辑框架}:实现无需针对每类编辑任务单独训练的通用编辑范式,显著提升模型在不同灾害类型、不同场景下的泛化能力和适应性。
    \item \textbf{高效的模型微调与适应策略}:通过参数高效的微调方法,降低新任务适配的计算和数据成本,推动大模型在遥感领域的实际部署和应用。
    \item \textbf{完善的评测基准与指标体系}:构建面向文本引导遥感图像编辑的评测基准和指标,填补该领域评测体系的空白,为后续相关研究提供标准化参考。
    \item \textbf{实证性性能提升}:在高分辨率、双时相RSCC灾害影像上,模型在图像质量、编辑准确性和语义一致性等方面均优于现有主流方法,推动遥感灾害图像智能生成与编辑技术的发展。
\end{itemize}

通过上述成果,项目将为灾害应急管理、科学研究和实际应用提供强有力的数据和技术支撑,促进遥感智能生成与编辑方法在灾害场景下的落地与推广。



\section{参考文献}
\printbibliography

\end{document}
